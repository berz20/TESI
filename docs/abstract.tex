\documentclass{report}
\usepackage[italian]{babel}
\usepackage[utf8]{inputenc}
\usepackage{graphicx}
\title{TESI}

\begin{document}

\begin{abstract}
%   Questa tesi nasce con l'obiettivo di realizzare un apparato sperimentale per lo studio dell'\textit{effetto Zeeman} su difetti reticolari allo stato solido. \\ 
%   Il sistema fisico utilizzato per lo sviluppo e la validazione dell’apparato energetico è il complesso azoto-vacanza in diamante. \\
%   Il \textit{centro NV} è un difetto puntuale in cui un atomo di carbonio nel reticolo cristallino viene sostituito da un atomo di azoto [\textit{N}] e un sito reticolare adiacente viene lasciato vuoto [\textit{V}].
%   L'unicità di tale struttura risiede nella possibilità di manipolazione a temperatura ambiente attraverso fattori esterni quali campi elettromagnetici e microonde, che risultano in risonanze nell'intensità della \textit{fotoluminescenza} del campione. 
%   Dal punto di vista energetico si hanno un \textit{ground-state} ed un \textit{excited-state}, entrambi tripletti di spin, oltre a due singoletti intermedi. \\
%   Le proprietà di spin del complesso lo rendono un \textit{magnetometro} naturale, infatti l'applicazione di un campo magnetico esterno comporta, per effetto Zeeman, frequenze di risonanza differenti tra le transizioni dallo lo stato a spin nullo a quello di \textit{spin up} e di \textit{spin down}. \\
%   Per visualizzare le risonanze del campione si adopera la tecnica \textit{ODMR} che consiste in una combinazione dell'\textit{ESR}, \textit{electronic-spin-resonance}, e di una metodologia di misura ottica, nel caso in questione relativa alla fotoluminescenza del campione, in modo da avere un \textit{pairing} tra frequenze \textit{MW} e campo magnetico. \\ 
%   Le caratteristiche di tale difetto reticolare sono state studiate utilizzando un microscopio confocale in fotoluminescenza ad alta sensibilità, realizzato appositamente nel corso dell'attività di Tesi. \\
%   Il campione in esame è consistuito in un cristallo in diamante contenente un ensemble ad elevata densità di centri NV. L’analisi in fotoluminescenza è stata svolta utilizzando un diodo laser a lunghezza d’onda di eccitazione di $\lambda = 520 nm$,appositamente assemblato, verificando che lo spettro, analizzato mediante un monocromatore, corrispondesse effettivamente a quello del cristallo interessato, confrontando le posizioni relative alla \textit{ZPL} ed alla \textit{banda fononica}. \\ 
%   L'effetto Zeeman permette di quantificare l'effetto del campo magnetico esterno sul \textit{centro NV}. Per poterne prendere visione si è sfruttata la tecnica \textit{ODMR}, posizionando il campione di diamante al di sopra di un'antenna ottimizzata che permettesse una scansione in frequenze, regolate da un applicativo esterno. \\
%   Data la configurazione del reticolo cristallino del diamante e delle struttura dell'\textit{INGV} si evince come, date le possibili quattro orientazioni, l'effetto del campo solenoidale non si limiti ad una singola proiezione su un asse bensì presenti quattro componenti distinte visualizzabili nello spettro ODMR. Si ottiene dunque una tecnica particolarmente precisa per misurare vettorialmente il campo magnetico circostante al campione. 
%   Sono state perciò effettuate numerose prese dati sia per ricavare sperimentalmente il \textit{fattore di Landé} del sistema, sia per evincere l'effettiva precisione magnetometrica dell'apparato confrontandola con una sonda di hall. \\
%   Dalla letteratura il \textit{centro NV} presenta un asse di splitting Zeeman a  $\nu=2.87 \ GHz$, ma ciò è strettamente applicabile a complessi il cui asse risulta perfettamente allineato con quello della radiazione, mentre nella situazione in analisi , date le molteplici orientazioni, le risonanze visualizzate nello spettro, che risultano essere la proiezione dell'effetto di splitting dello specifico sistema lungo il proprio asse, subiranno una traslazione presentando $\nu$ differenti. Dunque da questo shifting è possibile estrapolare in prima approssimazione l'angolo tra il singolo complesso ed il vettore di campo magnetico. \\
% Infine, l’impiego di un apparato di misura (eccitazione laser, generazione di  microonde) dotato di un’elettronica di controllo con risoluzione temporale dell’ordine del nanosecondo fornisce in prospettiva la possibilità di operare un controllo coerente con eccitazione ottica sullo stato di spin di un ensemble di difetti otticamente attivi, con importanti applicazioni future nel campo della sensoristica e del processamento dell’informazione quantistica.
% Dunque ricavando lo \textit{splitting} tra le due 
%  risonanze citate è possibile estrapolare il modulo del campo magnetico applicato lungo l'asse
%  dello specifico centro. Date le e iterando il procedimento su tutte le quattro coppie di risonanze,
%  relative alle possibili orientazioni del centro NV ottenere il vettore complessivo del campo. \\ 
  %%%%%%%%%%%%%%%%%%%%%%%%%%%%%%%%%%%%%%%%%%%%%%%%%%%%%%%%%%%%%%%%%%%%%%%%%%%%%%%%
Questa tesi nasce con l'obiettivo di realizzare un apparato sperimentale per lo studio
dell'\textit{effetto Zeeman} su difetti reticolari allo stato solido. \\ 
Il sistema fisico utilizzato per lo sviluppo e la validazione dell’apparato energetico
è il complesso \textit{azoto-vacanza} in diamante. \\
Il \textit{centro NV} è un difetto puntuale in cui un atomo di carbonio nel reticolo
cristallino del diamante viene sostituito da un atomo di azoto [\textit{N}] e un sito reticolare
adiacente viene lasciato vuoto [\textit{V}].
L'unicità di tale struttura risiede nella possibilità di manipolazione a temperatura
ambiente attraverso fattori esterni quali campi elettromagnetici e microonde,
che risultano in risonanze nell'intensità della \textit{fotoluminescenza} del campione. 
Le proprietà di spin del complesso lo rendono un \textit{magnetometro} naturale,
infatti, grazie allo schema energetico, composto da due tripletti di spin e due 
singoletti intermedi, l'applicazione di un campo magnetico esterno comporta, per effetto Zeeman,
frequenze di risonanza differenti tra le transizioni 
$[m_s = 0 \rightarrow m_s = +1]$ e $[m_s = 0 \rightarrow m_s = -1]$, con $m_s$ numero quantico di spin. \\
Le caratteristiche di tale difetto reticolare sono state studiate utilizzando
un microscopio confocale in fotoluminescenza ad alta sensibilità,
realizzato appositamente nel corso dell'attività di Tesi. \\
Il campione in esame è costituito in un cristallo in diamante contenente un \textit{ensemble}
ad elevata densità di centri NV. La validazione della risposta \textit{PL} è stata svolta utilizzando
un diodo laser a lunghezza d’onda di eccitazione di $\lambda = 520 nm$,
appositamente assemblato. Lo spettro, analizzato mediante un monocromatore,
si è dimostrato coerente con le posizioni, riportate in letteratura,
relative alla \textit{ZPL} ed alla \textit{banda fononica}. \\ 
Per l'analisi delle peculiarità magnetiche il campione è posizionato su
un antenna ottimizzata che permette la scansione in frequenze.
Si adopera la tecnica \textit{ODMR} che consiste in una combinazione dell' \textit{ESR},
\textit{electronic-spin-resonance}, e della \textit{PL},
analisi ottica della fotoluminescenza. Data la configurazione del reticolo cristallino
del diamante e delle struttura dell'\textit{INGV}
si evince come, date le possibili quattro orientazioni,
l'effetto del campo solenoidale, espresso dallo \textit{splitting} tra le coppie di risonanze,
presenti quattro componenti distinte visualizzabili nello spettro ODMR.
Si ottiene dunque una tecnica particolarmente precisa per misurare vettorialmente
il campo magnetico circostante al campione. \\ 
Il metodo di rilevazione magnetometrica sopra citato si è dimostrato attendibile,
come conferma il paragone con i dati di una \textit{sonda di Hall}. \\  
È stato osservato
Dalla letteratura il \textit{centro NV} presenta un asse di splitting Zeeman a  $\nu=2.87 \ GHz$,
ma ciò è strettamente applicabile a complessi il cui asse risulta perfettamente allineato
con quello della radiazione, mentre nella situazione in analisi , date le molteplici orientazioni,
le risonanze visualizzate nello spettro subiscono una traslazione presentando
$\nu$ differenti. Dal confronto con il modello teorico è stato ricavato l'effettivo angolo presente
tra gli assi del reticolo cristallino ed il campo magnetico esterno. \\
Infine, l’impiego di un apparato di misura (eccitazione laser,
generazione di  microonde) dotato di un’elettronica di controllo con risoluzione
temporale dell’ordine del nanosecondo fornisce in prospettiva la possibilità
di operare un controllo coerente con eccitazione ottica sullo stato di spin
di un ensemble di difetti otticamente attivi, con importanti applicazioni
future nel campo della sensoristica e del processamento dell’informazione quantistica.
\\\\\\\\\\\\\\\\\\\\\\\\\\\\\\\\\\\\\\\\\\\\\\\\\\\\\\\\\\\\\\\\\\\\\\\\\\\\\\\\

Questa tesi nasce con l'obiettivo di realizzare un apparato sperimentale per lo studio
dell'\textit{effetto Zeeman} su difetti reticolari allo stato solido. \\ 
Il sistema fisico utilizzato per lo sviluppo e la validazione dell’apparato energetico
è il complesso \textit{azoto-vacanza} in diamante, difetto puntuale dato dall'inserimento di un atomo di azoto nel reticolo cristallino del diamante. \\
Le proprietà di spin del complesso lo rendono un \textit{magnetometro} naturale,
infatti, grazie allo schema energetico, composto da due tripletti di spin e due 
singoletti intermedi, l'applicazione di un campo magnetico esterno comporta, per effetto Zeeman,
frequenze di risonanza differenti tra le transizioni da stati a spin nullo verso stati a spin up e down. \\
% $[m_s = 0 \rightarrow m_s = +1]$ e $[m_s = 0 \rightarrow m_s = -1]$, con $m_s$ numero quantico di spin. \\
Le caratteristiche di tale difetto reticolare si studiano utilizzando
un microscopio confocale in fotoluminescenza ad alta sensibilità,
realizzato appositamente nel corso dell'attività di Tesi. \\
Il campione in esame è costituito in un cristallo in diamante contenente un \textit{ensemble}
ad elevata densità di centri NV. La validazione della risposta \textit{PL} si svolge utilizzando
un diodo laser a lunghezza d’onda di eccitazione di $\lambda = 520 nm$,
appositamente assemblato. Lo spettro, analizzato mediante un monocromatore,
si dimostra coerente con le posizioni, riportate in letteratura,
relative alla \textit{ZPL} ed alla \textit{banda fononica}. \\ 
Per l'analisi delle peculiarità magnetiche il campione è posizionato su
un antenna ottimizzata che permette la scansione in frequenze.
Si adopera la tecnica \textit{ODMR} che consiste in una combinazione dell' \textit{ESR},
\textit{electronic-spin-resonance}, e della \textit{PL},
analisi ottica della fotoluminescenza, in modo da avere un \textit{pairing}
tra frequenze \textit{MW} e campo magnetico.  Data la configurazione del reticolo cristallino
del diamante e delle struttura dell'\textit{INGV}
si evince come, date le possibili quattro orientazioni,
l'effetto del campo solenoidale, espresso dallo \textit{splitting} tra le coppie di risonanze,
presenti quattro componenti distinte visualizzabili nello spettro ODMR.
Si ottiene dunque una tecnica particolarmente precisa per misurare vettorialmente
il campo magnetico circostante al campione. \\ 
Tale metodo di rilevazione magnetometrica si dimostra attendibile,
come conferma il paragone con i dati di una \textit{sonda di Hall}. \\  
Dalla letteratura il \textit{centro NV} presenta un asse di splitting Zeeman a  $\nu=2.87 \ GHz$,
ma ciò è strettamente applicabile a complessi il cui asse risulta perfettamente allineato
con quello della radiazione, mentre nella situazione in analisi , date le molteplici orientazioni,
le risonanze visualizzate nello spettro subiscono una traslazione presentando
$\nu$ differenti. Dal confronto con il modello teorico è stato ricavato l'effettivo angolo presente
tra gli assi del reticolo cristallino ed il campo magnetico esterno. \\
Infine, l’impiego di un apparato di misura (eccitazione laser,
generazione di  microonde) dotato di un’elettronica di controllo con risoluzione
temporale dell’ordine del nanosecondo fornisce in prospettiva la possibilità
di operare un controllo coerente con eccitazione ottica sullo stato di spin
di un ensemble di difetti otticamente attivi, con importanti applicazioni
future nel campo della sensoristica e del processamento dell’informazione quantistica.

\end{abstract}
\end{document}

