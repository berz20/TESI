\documentclass{report}
\usepackage[italian]{babel}
\usepackage[utf8]{inputenc}
\usepackage{graphicx}
\title{TESI}

\begin{document}

% \maketitle
% Questa tesi affronta tre diverse realizzazioni di un sistema elettronico veramente bidimensionale (2DES), stabilito sulla superficie dei semiconduttori elementari, ovvero Pt/Si(111), Au/Ge(111) e Sn/Si(111).
% Le caratteristiche delle strutture atomiche sulle superfici sono state studiate utilizzando la microscopia a tunneling a scansione e la diffrazione elettronica a bassa energia con particolare attenzione alla deposizione di Pt su Si(111). L'ispezione topografica rivela che gli atomi di Pt si agglomerano come trimeri, che rappresentano l'elemento costitutivo strutturale dei domini di sfasamento. Sorprendentemente, ogni trimmer viene ruotato di 30° rispetto al substrato, il che si traduce in un'inaspettata rottura della simmetria. A sua volta, questo rappresenta un esempio unico di struttura chirale su una superficie di semiconduttore e contrassegna Pt/Si(111) come un candidato promettente per processi catalitici su scala atomica. Le interazioni spin-orbita (SOI) svolgono un ruolo significativo su superfici che coinvolgono adatomi pesanti. Di conseguenza, si può osservare un aumento della degenerazione dello spin negli stati elettronici, definito "effetto Rashba". Un sistema candidato per mostrare tale fisica è Au/Ge(111). Si suggerisce che il suo grande foglio di Fermi esagonale sia diviso in spin mediante calcoli all'interno della teoria del funzionale della densità. Il chiarimento sperimentale si ottiene sfruttando le capacità uniche del rilevamento dello spin tridimensionale nella spettroscopia fotoelettronica con risoluzione di spin e angolo. Oltre alla verifica della divisione dello spin, è stato dimostrato che le componenti nel piano dello spin possiedono un carattere elicoidale, mentre si osserva anche una rotazione prominente fuori da questo piano lungo sezioni rettilinee della superficie di Fermi. Sorprendentemente e per la prima volta in un 2DES, vengono rivelate ulteriori rotazioni nel piano dello spin vicino a direzioni ad alta simmetria. Questo modello di spin complesso deve provenire da anisotropie cristalline ed è meglio descritto aumentando il modello Rashba originale con termini SOI simili a Dresselhaus di ordine superiore.
% L'uso alternativo di adatomi di gruppo IV con una copertura significativamente ridotta cambia drasticamente le proprietà di base di un 2DES. La localizzazione degli elettroni è fortemente migliorata e le caratteristiche dello stato fondamentale saranno quindi dominate dagli effetti di correlazione. Sn/Si(111) viene esaminato a tal proposito. Serve come realizzazione ideale di un reticolo triangolare, che soffre intrinsecamente di frustrazione di rotazione. Di conseguenza, l'ordine magnetico a lungo raggio è vietato e si presume che lo stato fondamentale sia un isolante antiferromagnetico a spirale (AFM) o un liquido di rotazione. Qui, la funzione spettrale della singola particella viene utilizzata come una quantità fondamentale per affrontare la complessa interazione di frustrazione geometrica e correlazioni elettroniche. In particolare, ciò si ottiene combinando i punti di forza complementari dei calcoli di approssimazione della densità locale (LDA) ab initio, la spettroscopia fotoelettronica con risoluzione angolare all'avanguardia e la sofisticata LDA+DCA a molti corpi. In questo modo si svela l'evoluzione di una "banda d'ombra" e di una banda che si ripiega all'indietro incompatibile con un ordine AFM a spirale. Inoltre, i processi di salto del vicino più vicino sono cruciali qui e le caratteristiche spettrali devono essere attribuite a uno stato fondamentale AFM collineare, contrariamente alle aspettative comuni per un reticolo di spin frustrato.
% In questa relazione di laboratorio si studiano le caratteristiche dei raggi cosmici incidenti sulla superficie terreste, con un particolare riguardo per il flusso e la velocità di questi ultimi.
% %La seguente esperienza ha il fine di rivelare e studiare le caratteristiche dei raggi cosmici che incidono sulla superficie terrestre. In particolare misurandone il flusso e la velocità.
%
% Per la prima fase si dispone di quattro rivelatori a fotomoltiplicatore, dei quali si impostano i valori di alta tensione e di \textit{threshold} che ne permettono un funzionamento adeguato, e successivamente si misura sperimentalmente l'efficienza. Per uno dei quattro si analizza inoltre la variazione dell'efficienza in funzione sia della tensione di lavoro sia della tensione di soglia. In questo modo, si ricavano le configurazioni di lavoro ottimali.
%
% Si estrapola quindi il \textit{rate} di eventi dalla curva di coincidenza, ottenuta utilizzando i moduli di DELAY, AND e OR come illustrato nei paragrafi successivi. %per discriminare le coincidenze dovute al passaggio di raggi cosmici da quelle accidentali,
%
% Successivamente si adopera una \textit{cosmic box}, apparato caratterizzato da dimensioni ridotte che lo rendono particolarmente portatile.
% %Un'altra presa dati è svolta, invece, attraverso un diverso tipo di rivelatore, una \textit{cosmic box}, apparato portatile.
% Grazie alle sue peculiarità si esaminano l'attenuazione dei raggi cosmici dovuta all'interazione con materiali lungo la loro traiettoria, la loro distribuzione angolare e l'accettanza del dispositivo stesso. 
% Quest'ultima sezione viene svolta con l'ausilio di una simulazione Monte Carlo.
%
% Altro obiettivo dell'esperienza è la misura della velocità delle particelle, che ci si aspetta vicina a quella della luce. A questo scopo, si sfrutta un modulo TAC: questo viene prima calibrato, poi utilizzato con i quattro scintillatori grandi per misurare il tempo di volo dei muoni cosmici.
%
% Dalle misure effettuate, si ricavano valori quali il coefficiente di attenuazione, l'energia media delle particelle, il flusso dei raggi cosmici e la loro velocità. 
% Le quantità sperimentali risultano in tutti i casi consistenti con i valori attesi, ad eccezione del flusso, che è inferiore al flusso medio dei raggi cosmici al livello del mare teorico: si ipotizza che la discrepanza sia dovuta prevalentemente all'impostazione di tensioni di soglia di grande ampiezza. 
%
%a diversi valori della distanza tra coppie di scintillatori posti sullo stesso piano.\\ I risultati della misura...

\begin{abstract}
  Questa tesi nasce con l'obiettivo di realizzare un setup sperimentale in grado di controllare coerentemente un sistema quantistico a due livelli a temperatura ambiente. Il sistema energetico in questione è quello del complesso \textit{azoto-vacanza} in diamante. 
  Le caratteristiche di tali strutture sono state studiate utilizzando un microscopio confocale, realizzato appositamente e modificato attraverso l'utilizzo di  filtri e particolari lenti in modo da massimizzare la visibilità per l'ambiente di lavoro.
  In primo luogo è stata osservata l'effettiva risposta del diamante in questione ad una stimolazione laser di $\lambda = 505 nm$, verificando che lo spettro, analizzato mediante un monocromatore, corrispondesse effettivamente a quello del cristallo interessato, confrontando le posizioni relative alla \textit{ZPL} ed alla \textit{banda fononica}. 
  L'effetto Zeeman permette di quantificare l'effetto del campo magnetico esterno sul \textit{centro NV}. Per poterne prendere visione si è sfruttata la tecnica \textit{ODMR}, posizionando il campione di diamante al di sopra di un'antenna ottimizzata che permettesse una scansione in frequenze, regolate da un apllicativo esterno. 
  Data la configurazione del reticolo cristallino del diamante e delle struttura dell'\textit{NV} si evince come, date le possibili quattro orientazioni, l'effetto del campo solenoidale non si limiti ad una singola proiezione su un asse bensì presenti quattro componenti distinte visualizzabili nello spettro ODMR. Si ottiene dunque una tecnica particolarmente precisa per misurare vettorialmente il campo magnetico circostante al campione. 
  Sono state perciò effettuate numerose prese dati sia per ricavare sperimentalmente il \textit{fattore di Landé} del sistema, sia per evincere l'effettiva precisione magnetometrica dell'apparato confrontandola con una sonda di hall.
  La risposta del sistema ha inoltre mostrato un comportamento primariamente inatteso all'aumentare dell'intensità del campo magnetico. Infatti dalla letteratura e dalle conferme sperimetnali il \textit{centro NV} presenta un asse di splitting Zeeman a  $\nu=2.87 \ GHz$, ma ciò è strettamente applicabile a complessi il cui asse risulta perfettamente allineato con quello della radiazione, mentre nel nostro caso, date le molteplici orientazioni, le risonanze visualizzate nello spettro, che risultano essere la proiezione dell'effetto di splitting dello specifico sistema lungo il proprio asse, subiranno una traslazione presentando $\nu$ differenti.
  Infine grazie alla specifica \textit{PCB} alla quale il diodo laser è collegato, è possibile regolarne l'emissione ottenendo una radaizione impulsata. Mediante una sequenza impulsata basata sul metodo \textit{Rabi} si è proceduto con l'analisi del campione. Lo scopo in tale contesto sarebbe stato di ottenere il valore della \textit{frequenza di Rabi} e da quest'ultima effettuare degli impulsi di \textit{spin echo} andando effettivamente a preparare il sistema energetico nello stato desiderato. A causa di limiti sperimentali, dovuti in primo luogo ad un apparato estremamente sensibile al rumore luminoso dell'ambiente e ad un laser che presenta un periodo di stabilizzazione ben superiore alla durata degli impulsi necessari, non è stato possibile visualizzare le oscillazioni di Rabi ed ovviamente svolgere i passaggi sopracitati. 


\end{abstract}
\begin{table}[]
    \centering
    \begin{tabular}{c|c|c|c}
        &    $\Delta{\nu}$ [MHz] &   $B =\frac{ h \ \Delta{\nu}}{2 \ g_e \ \mu_B} [mT]$ &   $B_{teslametro}$ [mT]   \\ \hline
    $\Delta{\nu}_1$   &   82  &   1.46    & 4.39  \\ 
    $\Delta{\nu}_2$   &   57  &   1.02    & 4.17  \\
    $\Delta{\nu}_3$   &   30  &   0.54    & 4.01  \\
    \end{tabular}
    \caption{Magnete 3 cm}
    \label{tab:my_label}
\end{table}

% \begin{figure}[!h]
% \begin{centering}
%     \hspace{-0.5cm}
%     \includegraphics[width=15cm]{foto/1843.png}
%     \caption{Magnete 5 cm}
%     \label{fig:zeeman_at}
% \end{centering}
% \end{figure}
\begin{table}[]
    \centering
    \begin{tabular}{c|c|c|c}
        &    $\Delta{\nu}$ [MHz] &   $B =\frac{ h \ \Delta{\nu}}{2 \ g_e \ \mu_B} [mT]$ &   $B_{teslametro}$ [mT]   \\ \hline
    $\Delta{\nu}_1$   &   67  &   1.20    & 2.30  \\ 
    $\Delta{\nu}_2$   &   57  &   0.86    & 2.02  \\
    $\Delta{\nu}_3$   &   30  &   0.18    & 1.80  \\
    \end{tabular}
    \caption{Magnete 5 cm}
    \label{tab:my_label}
\end{table}
\end{document}

